\chapter{Reading network interactions throughout development}

\epigraph{}{}

\section{Motivation}

Learning to reaada tunes neuronal sensitivity to stimuli. 

1. Interactive specialization posits that networks commonly accessed will become more refined. 
	a. More experience equates to a moer "coherent" structure. 
	b.
2. Studies have looked at wat refined means in terms of activation, but not in connectivity - during reading.
	a. Has not been studied often in reading comprehension.
3. There's an opportunity to investigate how reorganization changes with experience. 
	a. We might expect that there is less of a sneseory load as it becomes more automated. 
	b. 
4. 

There were two primary motivations in this study:

\begin{itemize}
	\item Which areas "converge" and "diverge" throughout development?
	\item What shifts in connectivity patterns do we see?
\end{itemize} 

\section{Methods}

\subsection{Participants}

\begin{table}
	\renewcommand{\tabcolsep}{0.09cm}
	\centering
	\begin{tabular}{lc}
\toprule
Measure &               Value \\
\midrule
Subjects                        &              47 \\
Total scan runs                 &              88 \\
Mean age                        &    10.51 (0.31) \\
Sex                             &      24 M, 23 F \\
WASI Full-Scale IQ, Vocabulary  &    53.15 (8.69) \\
Test of Word Reading Efficiency &  105.00 (18.09) \\
\bottomrule
\end{tabular}
	\caption{Participant demographics for study 2.}
	\label{table:ch3-participants}
\end{table}

\subsection{Functional MRI task design}

The task design for this study paralleled that of Chapter 2. Subjects were presented up to four separate runs of a passage comprehension task. The task included two passage blocks ("PASS"), two baseline attention blocks ("ATTN") and a trailing resting-state block ("REST"). Each task was performed in either the visual or auditory modality.

The contents of the passages differed from those of the tasks discussed in chapter 2, although they were also balanced to the same third-grade reading level using Coh-Metrix. 

\subsection{Data acquisition and processing}

To collect subjects at different stages in development, participants in this study were drawn from multiple study and age groups. They fell into three categories:

\begin{itemize}
	\item A group of children (ages 8 to 10) were selected from the third wave of the longitudinal study described in chapter 2. 
	\item A group of adolescents (ages 11 to 14) were selected. These participants were part of a large, cross-sectional study on the cognitive components of reading.
	\item A group of adults (ages 18 to 30) were selected, largely from a population of university research assistants and graduate students.
\end{itemize}

Functional MRI data were acquired on a Philips Achieva 3T MR scanner under the previously described parameters. Data were pre-processed in FSL and the CONN Toolbox before being analyzed.

\subsection{Activation analyses}



\subsection{Connectivity analyses}


\section{Results}


\section{Discussion}


