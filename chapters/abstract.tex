\chapter{Abstract}

\begin{itemize}
	\item \textbf{What we don't know:} The role that non-linguistic systems play in reading, especially comprehension. Specifically, how executive, attention and sensory systems interact to support skilled reading.
	\item \textbf{Why we care:} Understanding the sub-processes that make reading unique will allow us to refine efforts to teach older readers and those who struggle with comprehension.
	\item \textbf{How we'll address it:} By analyzing interactions between brain networks during language use and across different reading levels, we will identify networks that are more involved in reading than current models would predict. 
\end{itemize}

%% Think of this as a teaching document.

It is now a well-established finding that the human brain has a small-world architecture, which is characterized by densely connected modules and a highly-connected hub network. It’s been hypothesized that this architecture, and especially the flexibility of the hub network, enables the flexibility of human cognition, and disruptions to the architecture have been implicated in psychiatric disorders. Tasks induce a reorganization of this architecture, though, such that “flexibility” of the network may be of primary importance to efficient reading. However, how individual differences in this flexibility relate to individual differences in cognition remains to be explored.

Reading comprehension is a skill that requires the integration of multiple cognitive mechanisms, and therefore, requires passing information between many different neural networks. It is also a skill that has been well-characterized behaviorally, and so it represents an ideal skill for evaluating the importance of network “flexibility” for fluent and skilled behavior.

Insert literature review…..Thus, the following questions remain….

In this dissertation, we will investigate the following questions:


Question 1: What aspects of network architecture are sensitive to individual differences in reading?
H1. Modularity is sensitive to reading skill
H2. Participation coefficient is sensitive to reading skill
H3. At specific networks, we see greater or less sensitivity to reading skill

Question 2: How do individual differences in reading and listening organization relate to network architecture in tasks?
H1. Task complexity (e.g. reading vs rest) requires a reorganization of networks, such that lower modularity is indicative of successful performance.


Question 3: How do individual differences in "intrinsic" network architecture relate to behavior and task flexibility?
H1.	A more diversely-connected hub network at rest will be related to greater reorganization during reading.
H2.	A more diversely-connected hub network at rest will be related to *more variability*across multiple tasks – listening, reading, and sensory attention tasks.

The measure of flexibility may be related directly the fronto-parietal network, an executive network shown to have highly variable connectivity across different tasks. However, it could also be more generally associated with hub regions or modularity. We also have the ability to characterize these findings across development.