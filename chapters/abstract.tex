\chapter{Abstract}

\begin{itemize}
	\item \textbf{What we don't know:} What differentiates reading from listening, besides decoding? Are they simply different inputs to the same "language" system?
	\item \textbf{Why we care:} Understanding the sub-processes that make reading unique will allow us to refine efforts to teach older readers and those who struggle with comprehension.
	\item \textbf{How we'll address it:} By analyzing interactions between brain networks during language use and across different reading abilities, we will identify networks that are more involved in reading than current models would predict. 
\end{itemize}

We know that the single most important function in reading is \textit{decoding} visual stimuli into linguistic representations of words. Mastering this process creates permanent changes to brain structure and function, as well as to behavioral responses to speech. The dominant "parallel" model of reading suggests that this symbol-to-speech mapping process is the principal difference between reading and listening; that after feeding input "up" from visual or auditory systems, the linguistic processes are the same. However, many children struggle to comprehend text even after becoming proficient decoders, and there are clear differences in the nature of the modalities, including grammar, motor processes (e.g. eye movements), and differences in contextual information available (e.g. pitch, tone, rhythm in listening; peripheral words, recursive glances in reading). Therefore, decoding is not the only process that makes reading different from listening, but it is not clear which other cognitive processes might most contribute to reading success or failure. 

The "parasitic" model of reading asserts that reading is a skill built on top of the linguistic process (Mattingly 1972). These higher level processes are important for longer forms of reading - passage reading, for example, which is more common in older readers. Although disentangling the contributions from other skills (working memory, attention, planning/organization) during reading is difficult for behavioral research, it is a question that is well-suited to tightly controlled neuroimaging studies. In particular, connectomics methods, which can test interactions between sub-networks of the brain, allow us to test cognitive models during tasks, to see which areas are most likely interacting during reading and listening. We test how modality-bound networks (auditory, visual), language networks (esp. default mode), and executive networks (attention and cognitive control) integrate and segregate during reading and listening, with the aim of identifying cognitive processes that are unique to reading.

A second test of the two models is whether these differences change or persist throughout development. There is strong evidence that learning to decode creates persistent changes to the neural systems utilized in language \citep{SeidenbergXXXX, DahaeneXXXX}, but whether differences due to other cognitive skills also induce changes is an open question. On the one hand, advocates of a "parallel" model might assert that additional cognitive processes are used to "bootstrap" reading skill at early ages, so that differences between the two systems should decrease with maturity and reading experience as the two systems merge. On the other hand, the "parasitic" model of reading would predict that differences would persist, or even become exaggerated, as the two systems become more efficient for their respective tasks. We test this by comparing network interactions during reading and listening throughout development (children, adolescents, adults).  

Finally, we turn to the question of whether some children are better equipped to handle the additional complexity of reading. In particular, we are interested in whether this  might be apparent through resting-state fMRI, which can be performed before children even start reading. As shown in the literature, we expect to see increases in "intrinsic" connectivity between secondary visual processing areas and language areas; however, the "parasitic" model would suggest that there would also be increased connectivity between the visual network and other systems throughout development and in better readers. 

Overall, this set of analyses will reveal the set of brain networks (and their interactions) that are important for reading, especially reading comprehension. The results will help tailor models of reading beyond the "simple view" that reading is simply "decoding plus speech comprehension". Rather, they suggest that while reading taps visual areas and engages core language systems, there are neural processes that occur more acutely during reading which differentiate it from listening: reading is its own process, although it "sits on top" of listening systems. Analyses across development and relating individual differences to differences in intrinsic network architecture provide further support for these claims that non-decoding-related neural connections are important for skilled reading. 
