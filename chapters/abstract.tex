\chapter{Abstract}

% Think of this as a teaching document.

It is now a well-established finding that the human brain has a ``small-world'' architecture, which is characterized by densely connected modules and a highly-connected hub network. It has been hypothesized that this organization is central to the immense flexibility of human cognition, and inefficiences in this architecture have been implicated in psychiatric disorders. 

Tasks induce a reorganization of this architecture, though, such that ``flexibility'' of the network may be of primary importance to skill in multiple cognitive domains. However, how individual differences in this flexibility relate to individual differences in cognition remains to be explored.

Reading comprehension is a skill that requires the integration of multiple cognitive mechanisms, and therefore, requires passing information between many different neural networks. It is also a skill that has been well-characterized behaviorally, and so it represents an ideal model ability for evaluating the importance of network flexibility for fluent and skilled behavior.

The overarching question that this dissertation seeks to address is, to what extent are global and RSN-level measures of function related to individual differences in reading? To address this question, we conduct four studies to investigate the following questions:

\begin{enumerate}
	\item What aspects of network architecture are sensitive to individual differences in reading?
	\begin{itmeize}
		\item Global metrics related to integration such as modularity, participation coefficient and path length will be predictive of reading skill. 
		\item Reading-related networks will have a higher predictive power than general RSNs.
	\end{itemize}

	\item How does the act of reading re-organize network architecture, and does it differ for better readers?
	\begin{itemize}
		\item Complex tasks such as reading will increase between-RSN connectivity, reducing global metrics of integration.
		\item Better readers will have higher between-RSN integration. 
	\end{itemize}

	\item Across a variety of tasks, is it advantageous to have high ``flexibility'' of global network states? 
	\begin{itemize}
		\item Similarity between listening and reading network architecture will correspond to higher reading skill.
		\item Dissimilarity between language comprehension tasks and other tasks will correspond to higher reading skill.
	\end{itemize}

	\item How does the relationship between network architecture and reading skill change across the lifespan?
	\begin{itemize}
		\item In children, efficient network architecture will be highly related to reading skill; in adults, the effect will be weaker.
		\item In adults, there will be less ``global reorganization'' of brain networks across a variety of tasks.
	\end{itemize}
\end{enumerate}

The measure of flexibility may be related directly the fronto-parietal network, an executive network shown to have highly variable connectivity across different tasks. However, it could also be more generally associated with hub regions or modularity. We also have the ability to characterize these findings across development.