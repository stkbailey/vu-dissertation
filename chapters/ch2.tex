\chapter{Intrinsic network architecture and its relationship to skilled reading}

\epigraph{Modularity in the organization of biological systems confers significant advantages in an evolutionary setting, by supporting adaptability and robustness and thus increasing the system's evolvability.}{Sporns \& Betzel\\\textit{Modular Brain Networks}}

\section{Motivation}

% Resting-state network background and individual differences
The organization of the brain into modular networks (RSNs) is central to its efficient small-world organization. RSNs are distinct over time and composed of brain regions that tend to function as a unit \citep{DeLuca2006, Smith2009, Yeo2011}. This functional subdivision of the whole-brain network is hypothesized to provide a neural substrate for the diversity of cognitive functions in which people engage \citep{Yeo2014}. While these networks generally are thought to be highly similar across people \citep{Damoiseaux2006}, individual variations in the intensity or spatial extent of RSNs have been noted. For example, differences have been observed in individuals who exhibit variation in cognitive ability \citep{Reineberg2015, Tian2013} and in individuals with genetic susceptibility to Alzheimer’s disease \citep{Filippini2009}. Moreover, the patterns of RSN activity within an individual is consistent over periods of time in excess of three years \citep{Choe2015}. It is thought that this stability may reflect a history of co-activation among brain regions that occurs over time \citep{Power2010}. Consequently, variation in RSNs are likely to be stable measures of individual differences. To date, however, their relationship to other cognitive domains, and in particular reading, is not well understood.  Given that we know broad scale networks underlie a variety of cognitive functions, it follows that individual variation within certain reading-related networks and the interaction among RSNs could be linked to varying reading abilities. The existence of these associations can be used to not only predict cognitive and academic functioning at the level of the individual, but also aid treatment allocation or prediction of developmental trajectory \citep{Mattfeld2014} or response-to-intervention \citep{Crowther2015, Whitfield-Gabrieli2015}. 

% Graph measures of connectivity
A major question is how best to model network properties so that they are both sensitive to network composition (e.g. default mode, fronto-parietal) and globally representative. For example, graph measures of network efficiency are global, but they do not reflect RSN properties; connectivity measures related to a region of interest such as the visual word form area, might be sensitive to differences in auditory-visual processing, but do not reflect global organization \citep{Rubinov2010}. Although whole-brain measures of network architecture architecture have been correlated with individual differences in intelligence and psychiatric disorders \citep{}, employing measures that are sensitive to regional variation, such as modularity and the participation coefficient, could better clarify specific drivers of individual differences. The downside is that these measures require a definite parcellation of the brain into RSNs, which was not possible in the formative years of this field, when exploratory research was first identifying and describing the RSNs. 

% Global measures, the reading system
In general, skilled reading is associated with left hemisphere language and word recognition regions (left inferior frontal, supramarginal and occipito-temporal regions), and fronto-parietal regions supporting attention \citep{Paulesu2014}. Understanding how their coordination at rest relates to variability in reading ability is of interest. The brain regions supporting reading do not, however, form a unique, fundamental network. Instead, reading appears to rely on the reconfiguration and integration of multiple, more fundamental brain networks \citep{Koyama2010, Vogel2013}. Indeed, emerging research suggests that functional connectivity indices are associated with differences in reading skill. Struggling readers, such as those with dyslexia, exhibit decreased connectivity between visual association areas and prefrontal attention areas, increased right hemisphere connectivity, and reduced connectivity to occipito-temporal cortex compared to non-impaired readers \citep{Finn2014}.  In typically developing readers, Koyama and colleagues found increased positive connectivity among language regions was associated with increased word reading ability \citep{Koyama2011}, and recent studies suggest that interventions designed to enhance reading skill can increase the correlations between visual and frontal executive areas \citep{Horowitz-Kraus2015}. It is therefore likely that fast and efficient readers have coordinated neural systems that include both traditional language areas, as well as broader, more domain-general networks, which might be reflected in global measures of network connectivity.

% RSN measures that might be useful
Although most cognitive functions are relevant to reading in some way, the RSNs most associated with primary reading subprocesses are the fronto-parietal, ventral attention, and visual RSNs.  In addition to the intensity of and specific spatial composition of these networks, coordination between these brain areas and others are also likely to play a role in differentiating higher from lower performing individuals. For example, Koyama and colleagues found that increased reading ability was associated with increased negative connectivity between reading regions and regions of the default mode network, a network typically implicated in internally-directed thinking \citep{Andrews-Hanna2011}. This negative relationship between the default and reading networks echoes work showing that increased anti-correlated activity between the default network and regions specialized for cognitive function, such as those involved in attention \citep{Kelly2008, Mennes2010, Seeley2007}, inhibitory control \citep{Tian2013}, and working memory \citep{Keller2015, Sala-Llonch2011} is associated with individuals who display higher performance. 

% Study setup
An in depth examination of how behavioral indices of reading relate to various properties of RSNs has not been previously reported. Therefore, in this chapter, we investigate what relationship, if any, intrinsic network architecture has with individual differences in reading. First, we validate the existence of a small-world architecture in these subjects, and that the network parcellation is appropriate for them. Second, we determine whether global measures of network architecture, including modularity, participation coefficient and path length, are related to reading skill. Finally, we investigate which RSNs drive the relationship between connectivity and reading skill. To address each of these questions, we analyze resting-state fMRI in developing readers, which has the potential benefit of being able to be performed before children even start reading.


\section{Methods}

The following methods detail the current study's protocol and analytic approach. The following chapters borrow heavily from the methods described below, so they are explained here in detail. 

\subsection{Participants}

Participants were drawn from the fourth wave of a larger, longitudinal study investigating the neurobiological bases of reading comprehension. In total, 52 children completed scans and a subset of these met the motion and attention thresholds described below.

All participants were native English speakers with normal hearing and normal or corrected vision, and no history of major psychiatric illness or traumatic brain injury/epilepsy. Subjects had no history of a developmental disability or contra-indication to MRI.  Each participant gave written consent at the beginning of the study, with procedures carried out in accordance with Vanderbilt University’s Institutional Review Board.

\begin{table}
    \renewcommand{\tabcolsep}{0.09cm}
    \centering
    \begin{tabular}{lc}
\toprule 
Measure & Subjects \\ 
\midrule 
No. Participants				& 42 \\ 
No. Scan Runs					& 164 \\ 
Gender  						& 25 F \\ 
Age at Scan 					& 10.5 (0.3)  \\ 
WASI Full-Scale IQ  			& 111.0 (16.2) \\ 
TOWRE - Total Word Efficiency 	& 104.6 (18.5) \\ 
\bottomrule 
\end{tabular}
    \caption[Participant demographics]{Demographics and mean test scores for Study 1 participants are described here. For continuous data, the standard deviation is enclosed in parentheses.}
    \label{table:ch2-participants}
\end{table}

In addition to having an MRI scan, participants completed cognitive tests, including the Wechsler Abbreviated Scale of Intelligence (WASI) \citep{Kaplan1999}, the Test of Word Reading Efficiency (TOWRE) \citep{Torgesen2012}, the Woodcock Reading Master Tests (WRMT) \citep{Woodcock1998}, and the Gates-MacGinitie Reading Comprehension test \citep{MacGinitie2000}. Demographics and selected test data are summarized in Table \ref{table:ch2-participants}.

\subsection{MRI acquisition and preprocessing}

Imaging was performed on a Philips Achieva 3T MR scanner with a 32-channel head coil. Functional images were acquired using a gradient echo planar imaging sequence with 40 (3 mm thick) slices with no gap. Resting-state fMRI scans consisted of 150 dynamic volumes. Slices were parallel to the anterior-posterior commissure plane. Imaging parameters for functional images included: TE = 30 ms; FOV = 240 x 240 x 120 mm\textsuperscript{3}; flip angle = 75\degree; TR = 2200 ms; and 3 mm\textsuperscript{3} isotropic voxels.

\begin{figure}[t]
    \centering
    \includegraphics[width=6in]{ch2-connectome-methods}
    \caption[Schematic for connectome construction.]{Connectomes are constructed from resting-state fMRI in the following steps: slice-timing correction, rigid-body motion correction, boundary-based registration to a T1-weighted anatomical image, and normalization to MNI 152 template. The timseries for 264 nodes are then extracted and denoised using signal from non-neural tissue, continuous motion parameters and outliers. A pair-wise connectivity matrix is then calculated and thresholded at multiple different thresholds, then analyzed. Figure adapted from \citep{Yang2018}.}
    \label{fig:ch2-connectome-methods}
\end{figure}

Whole-brain fMRI analyses were performed using tools from the FMRIB Software Library (version 5.0.9). For each session, the following pre-processing steps were performed:  slice-time correction, motion correction to the initial fMRI volume, boundary-based registration to the subject's structural image, and normalization to 2 mm MNI 152 standard space. To mitigate the effects of motion on our analyses, we regressed out 6 continuous motion parameters and scrubbed out outlier volumes. We defined an outlier volume as any in which the root-mean-square framewise displacement exceeded 0.7 mm. Because head motion can be a major confound for connectivity analyses, we removed scan runs where more than 20 percent of the fMRI volumes were outliers.

\subsection{Network construction}

To investigate whole-brain patterns of connectivity with minimal investigator bias, we selected 264 nodes \textit{a priori} whose connectivity properties have been extensively analyzed in previous works \citep{Power2011}. The node set samples the entire brain, and nodes were selected based on their involvement in a diversity of cognitive tasks. Each node was assigned to one of 13 RSNs based on previous literature \citep{Power2013}. Approximately 10 percent of the nodes did not have a stable assignment in the original paper; for the present analyses, these nodes were excluded from graph theory calculations. A description of the 13 networks and their sizes is provided in \ref{table:ch2-power-nodes}. 

\begin{table}[t]
	\renewcommand{\tabcolsep}{0.09cm}
	\centering
	\begin{tabular}{lcc}
\toprule 
Suggested RSN & Abbreviation & Nodes \\ 
\midrule 
\textit{Sensory} & & \\
	\hspace{3pt}Auditory  			&  AUD & 13 \\ 
	\hspace{3pt}Somatomotor (Hand)	&  SOH & 30 \\
	\hspace{3pt}Somatomotor (Mouth)	&  SOM & 5 \\
	\hspace{3pt}Visual	 			&  VIS & 31 \\ 
\textit{Attention} & & \\
	\hspace{3pt}Dorsal attention  	&  DAN & 11	\\ 
	\hspace{3pt}Salience		  	&  SAL & 18 \\ 
	\hspace{3pt}Ventral attention  	&  VAN & 9 \\ 
\textit{Executive} & & \\
	\hspace{3pt}Cingulo-opercular 	& CON & 14 \\ 
	\hspace{3pt}Default mode		& DMN & 58 \\
	\hspace{3pt}Fronto-parietal  	& FPN & 25 \\ 
	\hspace{3pt}Memory retrieval	& MEM & 5 \\
\textit{Other} & & \\
	\hspace{3pt}Cerebellar			& CER & 4  \\
	\hspace{3pt}Subcortical			& SUB & 13 \\
	\hspace{3pt}Not assigned 		& UNC & 28 \\ 
\bottomrule 
\end{tabular}
	\caption[List of networks.]{List of networks used in connectivity analyses and the number of nodes affiliated with each. Although alternative parcellations of the node set are possible, we elected to use those network assignments suggested in \citep{Power2013}.}
	\label{table:ch2-power-nodes}
\end{table}

Connectivity analysis was performed in the CONN toolbox \citep{WhitfieldGabrieli2012}. fMRI data were high-pass filtered at 0.008 Hz, motion-corrected, co-registered to a structural image, normalized to MNI space and smoothed by a 5 mm FWHM spherical kernel. Outlier volumes were identified as individual fMRI volumes in which the RMS framewise-displacement exceeded 0.7. fMRI timeseries were corrected using anatCompCorr methods, which uses signal from white matter tissue and cerebrospinal fluid areas to reduce noise not related to brain activity \citep{Chai2012}. We also regressed out 12 continuous measures of motion were also included and all outlier timepoints. The timeseries was then high-pass filtered at 0.01 Hz. fMRI timeseries correlations were calculated between each of the the 264 nodes, resulting in a single connectivity array for each subject at each time point. Matrices were then thresholded into binary maps by keeping the top 5 percent of connections. (To confirm that this particular threshold did not unduly influence results, we swept results between thresholds at the top 2 percent to the top 10 percent of connections. No significant effect on the results was found.)

\subsection{Network analyses}

The metrics of interest were network \textit{modularity}, \textit{participation coefficient} and \textit{path length}. Modularity is high in networks where nodes within the same RSN are highly connected to each other but not elsewhere. The participation coefficient, on the other hand, is high when many nodes are connected to several different RSNs. Both of these metrics relate to the integration of information between RSNs. Path length describes the distance between any two nodes on the graph. This was calculated between every node, then summed up by RSN to create a measure of network distance. These properties, and their changes within our task, were investigated at the level of connectomes, RSNs and nodes. 

First, we establish the validity of the parcellation for evaluting network properties in this sample. At rest, we expect to see high modularity (greater than 0.1), low participation (less than 0.9), and a lower path length within RSNs than between them. We also expect to see moderate-sized correlations between measures, since each is measuring an aspect of network architecture related to distance between nodes.  

Next, we break each global measure down by RSN to determine how sub-systems differ in their network roles. For modularity, we report the total modularity contribution for each network. For participation coefficient and path length, we report the mean value within each network. We also investigate the measures obtained across the whole range of network-forming thresholds (retaining the top 2 to 10 percent of connections). We expect to see changes in the measures across thresholds, but ranked in a relatively stable order among the different RSNs. 

To determine the relationship between network measures and individual performance on cognitive assessments, we input each global metric into a general linear model with the Test of Word Reading Efficiency (TOWRE, total word efficiency (TWE) standard score). Models containing measures of mean framewise-displacement (motion) and the WASI Vocabulary measure were also assesed to ensure that effects were not driven by motion confounds or global measures of cognitive skill. We also examined whether there were differences in the modularity relationship between TOWRE subtests (sight word efficiency (SWE) and phonemic decoding efficiency (PDE)). 

To assess whether there was an RSN-level trend in the modularity-to-reading relationship, post-hoc analyses comparing network-level modularity values to TOWRE scores were also investigated. For each node, a correlation value was calculated between its modularity contribution and TOWRE TWE scores. To evaluate significance, RSN correlations were compared to a bootstrapped distribution of 5000 correlation values generated by sampling and totaling the modularities for a random set of nodes equal in size to the selected RSN. (For example, adding 31 random nodes for comparison to the visual network.)


\section{Results} 

Of the 52 subjects who completed resting-state fMRI scans, 44 met the scan quality criteria for inclusion. Connectome parcellations at the 5 percent threshold exhibited small-world properties: the mean modularity value was 0.264 (SD = 0.037), and the mean participation coefficient was 0.599 (0.052). Furthermore, the path length within RSNs was significantly lower than those between: within-community nodes took an average of 2.49 (0.141) steps to reach each other, whereas between-community nodes took an average of 3.06 (0.207) steps ($p$ \textless 0.001, two-sample $t$-test). Furthermore, a comparison of each metric against the others shows that, while there is overlap between the measures at the global level, there is substantial variability as well (Table \ref{fig:ch2-global-graph-theory-descriptions}).

\begin{figure}[t]
    \centering
    \includegraphics[width=5in]{ch2-global-graph-theory-descriptions}
    \caption[Distribution and correlations between global graph theory measures.]{Distribution and correlations between the global modularity, participation coefficient and path length. Each attribute may be interpreted as a measure of connectedness between RSNs, but there is significant variability between them.}
    \label{fig:ch2-global-graph-theory-descriptions}
\end{figure}

Figure \ref{fig:ch2-network-graph-theory-descriptions} highlights the contributions of each RSN to the graph theory measures. The visual, somatomotor, and default mode were the most modular RSNs, in part reflecting their larger size relative to others. The dorsal attention, auditory and cingulo-opercular networks were the most participatory RSNs at rest. The fronto-parietal RSN occupied an interesting place, possessing relatively high modularity but also one of the higher participation coefficients and lower global path lengths. The effect of changing thresholds had consistent effects on each measure: as more connections were included, the modularity decreased, participation coefficient increased, and path length decreased (Fig.  \ref{fig:ch2-network-graph-theory-descriptions}, bottom). 

\begin{figure}[t]
    \centering
    \includegraphics[width=6in]{ch2-network-graph-theory-descriptions}
    \caption[Relationships between network-level graph theory measures.]{Relationships between network-level graph theory measures. Shown above are the network-level distributions of the graph theory measures when graphs are thresholded for the top 10 percent of connections (top row), and the network means as the network-forming thresholds are swept from 2 percent to 10 percent (bottom row).}
    \label{fig:ch2-network-graph-theory-descriptions}
\end{figure}

The relationships between each graph theory metric to TOWRE are summarized in Table \ref{table:ch2-global-glm-results}. Global modularity, but not participation coefficient or path length, was predictive of reading skill even after controlling for mean framewise displacement (motion) and verbal intelligence ($Z_{TWE} = 2.536$). The direction of the relationship was positive, and it was higher for the ``sight word efficiency'' subtest than for ``phonemic decoding efficiency'' ($Z_{SWE} = 2.779$, $Z_{PDE} = 2.138$), which did not reach significance when confounds were controlled. 

\begin{table}[t]
    \renewcommand{\tabcolsep}{0.09cm}
    \centering
    \begin{tabular}{lcc}
\toprule
Predictor   &  $Z$-statistic &  $p$-value \\
\midrule
\textsc{Test of Word Reading Efficiency} & & \\
\textit{Total Word Efficiency} & & \\
    \hspace{5pt}Modularity &  2.536 &  0.011* \\
    \hspace{5pt}Part. Coeff. & -0.339 &  0.505 \\
    \hspace{5pt}Path Length & -0.763 &  0.484 \\
\textit{Phonemic Decoding Efficiency} & & \\
    \hspace{5pt}Modularity  &  2.138 &  0.033* \\
    \hspace{5pt}Part. Coeff. & -0.268 &  0.479 \\
    \hspace{5pt}Path Length & -0.697 &  0.561 \\
\textit{Sight Word Efficiency} & & \\
    \hspace{5pt}Modularity  &  2.779 &  0.006** \\
    \hspace{5pt}Part. Coeff. & -0.418 &  0.547 \\
    \hspace{5pt}Path Length & -0.750 &  0.459 \\
\textit{Total Word Efficiency (controlling motion)} & & \\
    \hspace{5pt}Modularity  &  1.991 &  0.047* \\
    \hspace{5pt}Part. Coeff. &  0.089 &  0.600 \\
    \hspace{5pt}Path Length & -0.642 &  0.533 \\
\textit{Total Word Efficiency (controlling verbal IQ)} & & \\
    \hspace{5pt}Modularity  &  2.390 &  0.017* \\
    \hspace{5pt}Part. Coeff. &  0.167 &  0.693 \\
    \hspace{5pt}Path Length & -1.074 &  0.288 \\
\bottomrule
\end{tabular}
    \caption{Results for analyses comparing global graph theory metrics to reading skill.}
    \label{table:ch2-global-glm-results}
\end{table}

The relationship between modularity and TOWRE is stable across multiple thresholds (Fig. \ref{fig:ch2-global-glm-covariates-thresh}). In fact, when graph theory measures were compared to other language-related assessments, there was a trend towards a significant positive relationship between modularity and cognitive performance that was more stable than those of either the participation coefficient or global path length. 

\begin{figure}[t]
    \centering
    \includegraphics[width=5.5in]{ch2-global-glm-covariates-thresholds}
    \caption[Modularity metrics at rest are the best predictors of cognitive skills.] {Global modularity was the most stable and predictive network measure for predicting language-related skills, although it only reached significance thresholds for the TOWRE.}
    \label{fig:ch2-global-glm-covariates-thresh}
\end{figure}

Finally, we investigated the correlation between each individual RSN's modularity contribution and TOWRE scores (Fig. \ref{fig:ch2-rsn-node-modularity-corr}. Overall, no RSN reached the level of significance that the global modularity measure (i.e., $r = 0.378$). The default mode RSN had the highest correlation with the TOWRE ($r_{DMN} = 0.350$), with the memory retrieval ($r_{MEM} = 0.303$) and attention ($r_{DAN} = 0.236$, $r_{VAN} = 0.248$) RSNs ranking next. Although not significant, the relationship was inverted in the auditory RSN: lower modularity was associated with better reading skill ($r = -0.178$).

\begin{figure}[t]
    \centering
    \includegraphics[height=3in]{ch2-rsn-node-modularity-corr}
    \caption[Modularity relationship with TOWRE varies by RSN.] {Modularity relationship with TOWRE varies by RSN. Although no RSN reached the level of significance that the global modularity measure, the default mode and memory retrieval RSNs had significant positive correlations with the TOWRE.}
    \label{fig:ch2-rsn-node-modularity-corr}
\end{figure}

\section{Discussion}

For this first study, we sought to establish whether network measures of resting-state fMRI data were related to reading skill. We established a series of methods and measures for summarizing the global and RSN-level network architecture, and showed that of all metrics, modularity was most robustly able to index reading skill in our sample. We also explored how variations of this measure among RSNs is associated with individual differences in reading ability. This demonstrated that modularity within the default mode network is most similar to the modularity of the whole brain, and it also revealed significant negative correlations between the modularity of the auditory and cingulo-opercular networks and reading skill.

% Significance of graph theory descriptions
One advantage to our approach is that we used \textit{a priori} defined set of nodes and network parcellation. We then employed standard measures that are sensitive to RSN properties: modularity, participation coefficient and path length. Although adjusting the network-forming threshold had a sizeable impact on the values yielded for each metric, the relative contributions of each RSN to each metric were fairly stable. That is, there was a global and not local trend in the effect of thresholds, with a few exceptions for very small RSNs (for example, the memory retrieval network). Modularity, in particular, was relatively stable across thresholds, as indicated by its consistent relationship to cognitive metrics (Fig. \ref{fig:ch2-global-glm-covariates-thresh}).

% Significance of global findings
Modularity was the most effective measure for prediciting individual differences, especially in reading but more broadly in verbal skill. Modularity is a measure of network ``segregation'': the more different each RSN behaves during rest, the higher the modularity will be. This finding is consistent with previous literature showing that anti-correlations between the default mode network and the fronto-parietal network index cognitive skills \citep{Anticevic2012}. Although no consensus interpretation exists, one could speculate that increased negative correlations between the RSNs enables segregation of functions. This interpretation stems from a wide range of studies showing opposing activation in externally directed cognitive tasks for the frontoparietal and default RSNs, during which functions of the two networks ought to be segregated to prevent interference from the internal mentation functions of the default network. 

% Significance of RSN findings
We also investigated the relationship between the modularity contribution of each individual RSN with reading skill. Perhaps unsurprisingly, the default mode network was most highly correlated with reading and held the relationship most closely approximating the global modularity correlation. One interpretation is that the default mode modularity most closely approximates the global modularity, especially during rest. More interesting was the finding of an anti-correlation between modularity in the auditory and cingulo-opercular networks and reading skill. Although too coarse a measure to support this, the difference in modularity could be a marker of better readers' intrinsic connectivity with other systems: for example, visual processing areas.  


% Limitations of the present approach
While this study is one of the first to examine the relationship between reading skills and intrinsic network architecture, there are a few limitatis. One limitation is that we used a composite measure of reading. Since reading relies on a number of subordinate processes (e.g. word recognition and semantic processing), it is possible that our results do not reflect differences across specific domains relevant to reading and that such relationship might be observed. Another limitation is that, because we examined these relationships in relatively mature readers, it is possible that other relationships might be observed in developing readers. For example, our null findings for individual differences in the ventral attention RSN could reflect our sample’s relative maturity with respect to reading development since the ventral attention RSN overlaps with left hemisphere regions supporting lower-level reading processes such as orthographic processing. Such processes explain less and less variance in reading skill as texts become more difficult and reading starts to reach mature levels \citep{Cutting2006}. 

% Conclusions
Overall, the current results demonstrate that modularity is an important indicator of successful reading skill, and there appear to be regional variations which influence it. How this modular architecture changes during the reading process is an important question that we will tackle in the following chapter.

