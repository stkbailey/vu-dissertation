\textbf{Adjust the framing of Chapter 4 to reflect the preliminary nature of the findings. Specifically, add context to Chapter 1 description, Chapter 5 motivation and Chapter 6 review.)

I added the following sections to the document:

* Language concerning the "preliminary" nature of the fourth study in Chapter 1: "Study 4 provides a preliminary investigation of network topology in various evoked architectures across development."
* An additional statement about confounding of task difficulty and developmental effects in the introduction of Chapter 5: "We note that, although the data analyzed here parallel those described in Studies 1, 2 and 3, this also introduces some challenges. Most important of these is that developmental effects are confounded by the task difficulty and differences in motion. Even with these limitations, however, the results will provide important context for our previous findings and help to flesh out a model of the advantages conferred to specific cognitive skills by the modular brain."
* A paragraph on limitations in the conclusiosn of Chapter 5: "Although this study boasts several major features, including the replication of previous results, multiple cross-sectional age groups and a large number of analyzed scans (254 scans totalling nearly 30 hours of scan time), there are a few limitations. First of all, the passages were balanced to a third grade reading level, meaning that the younger groups may have found them challenging while the older groups found them very easy. While this is likely, we do not believe this unduly influences the results. The passages were presented at a pace that was slow enough for the young readers to follow along without difficulty, and they were also presented phrase-by-phrase so that older readers would still have to engage with the text for the same amount of time and retain a mental model over the same period. More pernicious may be the differences in scan motion between the two groups. However, we have included a narrower portion of the young population and also performed several layers of preprocessing to account for these effects."
* Preliminary language to the recap of Study 4 in Chapter 6: "Although these results require further validation with a larger sample size and better control of confounds such as text difficulty, they serve a useful preliminary role in supporting our developmental model of network architecture."

\textbf{Discuss the findings of Chapters 2 and 3 in a more theoretical context. Weave in the theoretical model more closely, leading up to the multi-task model.}

I have provided two sections where I elaborate more on the overarching model and how this model might intersect with other cognitive skills. 

* I have added a model and description for how cognitive and network models of reading may intersect, prior to diving into specifics: "The overarching model, shown in Figure 1.6, is that skilled reading requires the integration of many independent cognitive processes, which are distributed throughout the brain. These focal processing areas are nested within larger functional modules, and certain hub areas such as the inferior frontal gyrus and posterior superior temporal gyrus play crucial roles in actively binding reading-related information between these RSNs. This information can then be shared globally via the richly connected hub network, allowing for speedy updating of the situation model." The accompanying caption provides additional context: "Reading requires many independent cognitive processes, represented by the green (comprehension) and orange (word recognition) strands. From a neural standpoint, these processes are distributed and nested within RSNs (1). In reading, specific hub areas actively bind information between different RSNs, such as print and speech information (2). This information can then be shared globally via the richly connected hub network, allowing for speedy updating of the situation model (3)."
* In the conclusions of Chapter 2, I have added a paragraph in the discussion section focused on the intersection between reading and other abilities and suggest future directions for addressing the covariance between these skills."One important point to address in future studies is the specificity of these findings to reading skill. While modularity was the only attribute significantly predictive of TOWRE, it had a positive effect on WASI Vocabulary and Gates-Macginitie comprehension scores as well. This is unsurprising given the close relationship between oral language comprehension and word recognition measures such as the TOWRE \citep{Storch2002}. However, future studies will need to address the generalizability of these findings directly to determine to what extent these findings are specific to reading, given the diversity of relationships between reading, mathematics, executive function and even theory of mind \citep{Cantin2016}. One possibility is that attributes related to a high-functioning hub network will be generalizeable across many different skills (as in \cite{Bertolero2018}), whereas deficits in the functioning related to one or more specific hubs would manifest primarily in a single class of skills."

\textbf{Add measures of in-scanner task performance to Chapters 3 and 4.}

I have added the behavioral measures as in-text descriptions in the appropriate chapters.

* Chapter 3: 47 subjects (88 scan runs) met the attention and motion criteria for inclusion in the analysis. (5 subjects and 15 scan runs were excluded.) The distribution of performance and motion criteria are illustrated in Figure \ref{fig:ch3-task-performance}. Across all included scans, the mean frame-wise displacement was 0.143 (standard deviation of 0.091), and the mean $D^\prime$ score, which measures performance based on both active (correctly identifies repeats) and passive (few false alarm clicks) metrics, was 4.097 (1.296).  The mean score for the in-scanner comprehension task (in which subjects identified if a picture was related to the passage) was 1.67 (out of 2), and subjects missed both questions in only two of the 88 included scan runs.
* Chapter 4: 42 subjects (142 scan runs) met the attention and motion criteria for inclusion in the analysis. Across all included scans, the mean frame-wise displacement was 0.149 (standard deviation = 0.099), and the mean $D^\prime$ score was 3.964 (1.285).  The mean score for the in-scanner comprehension task (in which subjects identified if a picture was related to the passage) was 1.725 (out of 2), and only 4 of the 142 scan runs answering both questions incorrectly. Mean $D^\prime$ scores were unrelated to the modality of presentation (Mean $D^{\prime}_{A} = 3.678$, Mean $D^{\prime}_{V} = 4.006$, $p = 0.176$) 

\textbf{For analyses that warrant multiple-comparison correction, detail the procedure followed or justify lack of procedure.}

Figures and tables where multiple comparison (or lack thereof) needed to be addressed are listed below. I include their label, caption/title and a description of the changes I've made to address this area.
* Table 2.3: Tests have been reduced to reflect the post-hoc nature of the Sight Word Efficiency and Phonemic Decoding Efficiency analyses (only testing to explain a significant result.) The original analysis is significant at Bonferonni-corrected threshold ($p_{crit} = 0.017$). 
* Figure 2.5: Global Modularity at rest is the most stable metric: Multiple comparison corrections are not necessary here, as the result serves to illustrate the robustness against threshold effects for Table 2.3. This has been noted in the figure caption.
* Figure 2.6: Corrections were addressed using bootstrapped distribution sensitive to the size of each resting-state network. The two cited results are significant even with conservative Bonferonni correction ($p_{crit}=0.003$), and this has been noted in the results section.
* Figure 3.5: Reading induces a more integrated global network architecture. Added note stating that results are significant across Bonferonni correction for multiple comparisons ($p_{crit} = 0.0125$).
* Figure 3.6: RSN-level trends in task-evoked networks. These are intended to be post-hoc descriptive measures rather than statistical tests; p-values are used for reference. The caption around the figure has been updated.
* Figure 3.7: Rewiring diagram showing the changes in connectivity between networks during reading. The lines colored in the diagram are significant at an FDR-corrected level of 0.05, with the width increasing based on the effect size (number of connections). This has been added to the caption.

* Figure 3.4: Distribution of reading-related activity among connectome nodes, grouped by RSN. We settle for a descriptive analysis here rather than a complete test of the results: 

* Figure 4.4: Activation differences due to modality of presentation among RSNs, when both tasks were compared to their sensory baseline. 
* Figure 4.8: RSNs share a large degree of similarity. 
* Table 4.2: Correlation values between shared connectivity and cognitive skills.
* Figure 5.3: Developmental shifts in connectivity strength. The lines colored in the diagram are significant at an FDR-corrected level of 0.05, with the width increasing based on the effect size (number of connections). This has been added to the caption.


\textbf{Add tables and/or figures that describe "active hubs" in the four different task states.}

These were elective to add into the paper; I'll send these on Sunday. (Wanted to get this out today.)

\textbf{Add "chapters" header to Introduction}