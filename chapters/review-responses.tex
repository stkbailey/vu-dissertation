1. Adjust the framing of Chapter 4 to reflect the preliminary nature of the findings. Specifically, add context to Chapter 1 description, Chapter 5 motivation and Chapter 6 review.

A. I added "preliminary" language to the study description in Chapter 1 (page ___), an additional statement about confounding of task difficulty and developmental effects in the introduction of Chapter 5 (page ____), a limitations paragraph in the conclusiosn of Chapter 5 (page ____), and preliminary language to the recap of Study 4 in Chapter 6 (page ___).

2. Discuss the findings of Chapters 2 and 3 in a more theoretical context. Weave in the theoretical model more closely, leading up to the multi-task model.

I have added a model and description for how cognitive and network models of reading may intersect, prior to diving into specifics, in Chapter 1 (page...). In the conclusions of Chapter 2, I have added a paragraph in the discussion section focused on the intersection between reading and other abilities and suggest future directions for addressing the covariance between these skills. 

3. Add tables and/or figures that describe "active hubs" in the four different task states.

4. Add measures of in-scanner task performance to Chapters 3 and 4.

5. For analyses that warrant multiple-comparison correction, detail the procedure followed or justify lack of procedure.

The areas where multiple comparisons should be addressed are: 
* Table 2.3: 
* Figure 2.5:
* Figure 2.6: Bootstrapped distribution
* Figure 3.5: Different measures
* Figure 3.6: RSN-level in task-evoked networks.


6. Add "chapters" header to Introduction