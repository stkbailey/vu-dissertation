% OVerall
(1)	You use attention task to describe the visual/tone tasks.  I would stick with visual and tone, as attention is confusing (you’ve already got a lot of references to attention in the DAN, VAN, etc!)
(2)	Throughout the document, I think it would be very helpful for you to more carefully delineate when you are talking about within a RSN(s) versus globally.  In some sentences you talk about global modularity and then switch to modularity within the RSNs, and that is a lot to keep track of. So I would come up with a very clear way of describing each, and use the same terminology throughout.

(1) study #1: do the within modularity correlations take into account the # of ROIs in each RSN and how uneven they are distributed? Seems quite logical DMN would be highest correlated since it has the most ROIs?


%Study 2: 
I think you need a table listing the high participation coefficients at rest and whether each one was active during reading. This may offer a clue to one global question I have: why, given that all results point to much more similarity across tasks, can one have any specificity of impairment?  This will help you address this question, and help with something I would suggest you bring in the discussion: that nodes with high PCs, and whether/how they are used in tasks may reveal degree of and specificity of impairment. ... Which nodes are active in reading and also hub-like? So is this only node that has a high PC that is active during reading? I don’t think you need to correlate with out of scanner reading. Just which modes involved in reading task have high PCs (since you said most don’t).
A: We added a table.

% Study 3
(3)	So far, the only quite substantive thing is that Study 3 is a bit fuzzy and hard to follow. Also, you sort of “center stage” the FPN in setting up the motivation for study 3, and examining if it plays an important driving role. But I don’t see that taking shape in the approach or findings. 

(2) study #3: p. 68 is very confusing. I’m having trouble understanding what you did with this last analysis.

% Study 4
Study 4: p.80. Is this changes/differences between RC and rest?  And are you looking at coordination (correlation?) between RSNs (eg how FPN is correlating with VAN?).  I am not sure what you mean by “high degree of coordination”. ... It’s section 5.3, third paragraph and it refers to Figure 5.2 (maybe look at the version you sent to me to locate what I’m talking about). 

Study 4: Task difficulty... Regarding study 4 confound: you might be able to get at this issue by doing reading matched (vs age matched) comparisons. ...
A: We addressed as a limitation...

Re p. 68. Do you mean that across all the 3 tasks you calculated a similarity index, and so that the interpretation is that the visual network activated/coordinated most similarly across LC, RC, and rest, whereas VAN was the least similar? 



5. Chapter 5 in general: you need to be careful about referring to the study in this chapter as developmental as in longitudinal — it sounds like longitudinal quite often when you talk about it vs cross sectional 


% Study comments for studies 1-3:




1. Page 2 “it will be critical to go BEYOND creating” missing “beyond”
2. P.3 Woodcock not a great reference for individual diff in reading 
3. P4 —> “passage comprehension measures, on the other hand...”  the argument that imaging is somehow better because they are less variable is very tenuous, and I don’t think that claim can be made whatsoever. You refer to this same concept several times and I don’t see this as a valid argument for why this study/studies needs to be done. A better way to think of it is that it may help disentangle 4. 
P12-13, last sentence in paragraph before “Executive networks may support intervention response” section is very hard to parse/understand.
5. P. 19, Study 1: “...address the question of whether individuals....complexity of reading”. Do you mean how connectivity relates to reading skill?
6. P.22, the paragraph in 2.1 that starts with “A major question is how to best model...”  This seems like an overarching issue and like it belongs in the introduction section that discusses such issues.
7. P.31-32, end of section 2.3.  Was the correlation of -.178 actually the same for both auditory and Cingular-opercular networks??
8. P. 34 the bootstrapped values were 0-.4. That seems to suggest not very reliable results?? Also, the end of this paragraph needs something as seems like, so....? 
9. P.39 last paragraph before 3.2 you have a sentence with “along” when I think you mean “align with” or something like that.
10. I mentioned this before but you use “attention” and “sensory” for visual/tone control task. I would use those terms vs attention or sensory.
11. P.53 2nd to last paragraph at the end of Chapter 3 is very fuzzy/confusing - needs a re-write.
12. P.55 section 4.1 —> motivation isn’t very clear 13. P.58 last sentence before “The connectome across many tasks” section ends with “and”.
14. P.63 section 4.3 you talk about attention and comprehension measures - are these behavioral? I don’t understand what you mean in this sentence 15. P.67 sentence that starts with “Better readers did have a higher degree...” is missing a word.


1. P.75-76 “phonics instruction spends a great deal of time...” I think this is overstated and not something we know.
2. P.76 “We suggest that any effect of modularity...” why not both?
3. P. 80 you refer to DMN and salience as “higher order” as compared to younger readers using FPN. This doesn’t seem correct. You also refer to it in fig 5.2 
4. P.81 or number excluded in terms of poor readers?
