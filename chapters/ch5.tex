\chapter{Discussion}

The current study sought to determine whether aspects of the brain's network architecture are related to reading. The results suggest that an efficient network organization, i.e. one in which brain areas form clusters connected by hub regions, is important for skilled reading, and that dyslexia can be characterized by abnormal functioning of hub regions that map information between multiple systems. To our knowledge, this is the first time the relationship between modularity and hubness to reading skill has been described, adding to a foundation of work built on other connectivity methods.

A connectomics approach to reading illuminates -- not displaces -- previous neuroimaging research, much of which focused on localizing specific cognitive processes. One insight is that much of the "reading network" falls in domain-general RSNs such as the attention and executive networks (see Fig. 1). While these areas perform a specific function in reading, they are also often involved in other processes. For example, the dorsal attention network (DAN) encompasses the visual word form area, an area that has been the subject of much interest and debate in reading and dyslexia research \citep{McCandliss2003}. It is probable that this area is so important in reading not only because it is connected to language areas \citep{Bouhali2014}, but also because it is tightly tied to other areas that control goal-directed attention \citep{Vogel2014}. Koyama et al. (2013) found that children with a historical diagnosis of dyslexia had persistent de-coupling of the DAN compared to typical readers regardless of remediation status \citep{Koyama2013}. Vogel et al. (2012) found that reading ability in typical children and adults (including decoding and passage comprehension ability) predicted increased correlations between the visual word form area and the DAN \citep{Vogel2012}. The nesting of this orthographic-processing area within the DAN is thus important to its role in reading.

An efficient small-world organization in the resting brain requires a modular network architecture, which was tied here to better performance in reading. This relationship was particularly high in the visual, default mode, cingulo-opercular networks. It is not yet possible to say whether modularity within these specific RSNs correlates most highly with reading because of their functional roles in reading processes or whether they simply capture global trends better than other networks. There is some reason to suspect specificity, however. In studies of remediation-induced changes to connectivity, increased connectivity within the visual network \citep{Koyama2013} and cingulo-opercular network \citep{HorowitzKraus2015} have predicted reading improvement in dyslexic children. The default mode network, on the other hand, supports a wide range of cognitive processes important for comprehension, including theory of mind, narrative processing, and autobiographical recall \citep{Buckner2008, AbdulSabur2014}, and its cohesiveness during resting-state has been used to investigate other disorders \citep{Uddin2008}. Future work will need to examine not just the internal connectivity, but the relationships between these networks during reading and at rest. The default mode network, for example, is typically anti-correlated with "task-positive" networks such as the fronto-parietal network. A high degree of anti-correlation has been reported to be important for performance on a variety of cognitive processes \citep{Fox2005, Keller2015}, but recent work suggests that high modularity and connectivity of the default mode during higher-level cognition is fundamental to processes relying on self-referential and memory retrieval processes, such as those found in language \citep{Vatansever2015}. The dynamics behind these interactions will be important for further establishing a framwork for investigating the roles of specific networks during reading. 

The additional findings that hubs areas are key in dyslexia are not surprising: dyslexia has often been thought to be a disorder of combining information across different functional systems, and in the context of connectomics, hub areas play a privileged role in mediating information flow between RSN’s. For example, the posterior temporal sulcus connects visual and auditory networks by binding letters to sounds \citep{Blau2010, VanAtteveldt2009} and the inferior frontal gyrus has many different subdivisions supporting language parsing and manipulation \citep{Hagoort2005}. However, casting dyslexia dysfunction into a connectomics perspective opens up new hypotheses and research avenues. For example, the brain areas of interest and neuroimaging metrics can be unified across other developmental disorders, including ADHD, specific language impairment and autism \citep{Stam2014}. Another benefit is that it opens up many more avenues for investigating dyslexia using functional and diffusion MRI, which can be performed in younger children and without administering a cognitive task. 
s
\subsection{Dynamic modeling of network architecture}

We could use dynamic modeling....

\subsection{Individual parcellations}

A caveat with connectomics analyses, including those presented here, are that results for the modularity analyses are often based on RSN parcellations from previous literature \citep{Power2011}. It is possible that in this sample of children, there were differences in network organization that resulted in a lower global modularity but were in fact due to differences in organization (e.g. multiple sub-modules of the default mode network). The question of how network architecture develops over time, and how best to measure it is under active investigation \citep{Cao2016}. Its answer will have important implications for disentangling this complex interchange between development, network architecture and cognitive performance.

