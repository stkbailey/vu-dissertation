\chapter{Reading network interactions throughout development}

\epigraph{Development...}{Stanislas Dahaene}

\section{Motivation}

In the previous chapter, we established that reading does not appear to be "simply" a change in sensory systems. There are also more global changes in the degree of integration of different RSNs, especially those related to attention and executive control. However, this analysis was conducted in emerging readers: the participants were approximately 10 years old and were at a point in their education where basic skills were in place, but they were still relatively inexperienced readers.

It could be that the differences between the modalities were due to active support processes that were occuring between executive and attention regions, as the participants performed reading, a relatively novel task. On the one hand, advocates of a "parallel" model might assert that additional cognitive processes are used to "bootstrap" reading skill at early ages, so that differences between the two systems should decrease with maturity and reading experience as the two systems merge. On the other hand, the "parasitic" model of reading would predict that differences would persist, or even become exaggerated, as the two systems become more efficient for their respective tasks. In this chapter, then, we seek to replicate these results in a new cohort of subjects with a different set of passages, and then to investigate the effects of developmental maturity on the reorganizaton of brain networks during reading and listening.

There were two primary motivations in this study:

\begin{itemize}
	\item Which areas "converge" and "diverge" throughout development?
	\item What shifts in connectivity patterns do we see?
\end{itemize} 

\section{Methods}

\subsection{Participants}

To collect subjects at different stages in development, participants in this study were drawn from multiple study and age groups. They fell into three categories:

\begin{itemize}
	\item A group of children (ages 8 to 10) were selected from the third wave of the longitudinal study described in chapter 2. 
	\item A group of adolescents (ages 11 to 14) were selected. These participants were part of a large, cross-sectional study on the cognitive components of reading.
	\item A group of adults (ages 18 to 30) were selected, largely from a population of university research assistants and graduate students.
\end{itemize}

All subjects provided approval and were compensated according to our IRB.

\begin{table}[t]
	\renewcommand{\tabcolsep}{0.09cm}
	\centering
	\begin{tabular}{lc}
\toprule 
Measure & Subjects \\ 
\midrule 
No. Participants				& 42 \\ 
No. Scan Runs					& 164 \\ 
Gender  						& 25 F \\ 
Age at Scan 					& 10.5 (0.3)  \\ 
WASI Full-Scale IQ  			& 111.0 (16.2) \\ 
TOWRE - Total Word Efficiency 	& 104.6 (18.5) \\ 
\bottomrule 
\end{tabular}
	\caption{Participant demographics for study 2.}
	\label{table:ch3-participants}
\end{table}

\subsection{Functional MRI task design}

The task design for this study paralleled that of Chapter 2. Subjects were presented up to four separate runs of a passage comprehension task. The task included two passage blocks ("PASS"), two baseline attention blocks ("ATTN") and a trailing resting-state block ("REST"). Each task was performed in either the visual or auditory modality.

The contents of the passages differed from those of the tasks discussed in chapter 2, although they were also balanced to the same third-grade reading level using Coh-Metrix. 

\subsection{Data acquisition and processing}

Functional MRI data were acquired on a Philips Achieva 3T MR scanner under the previously described parameters. Data were pre-processed in FSL and the CONN Toolbox before being analyzed. See the Methods section of Chapter 2 for a detailed description of preprocessing routines and their parameters.

\subsection{Activation analyses}

We calculated subject-level contrast maps as described previously ("listening + reading vs. rest", "listening vs. reading"). At the group-level, we used age at scan as a continuous variable to estimate the effect of age on comprehensino-related activation and modality differences.

As before, we also investigated these effects in the connectome space as well 


\subsection{Connectivity analyses}


\section{Results}

\begin{figure}[t]
	\centering
    \caption[Relationship between activation in visual word form area and age.]{}
\end{figure}


\begin{figure}[t]
	\centering
    \caption[Global participation coefficient as a function of age.]{}
\end{figure}


\begin{figure}[t]
	\centering
    \caption[Similarity between reading and listening networks as a function of age.]{}
\end{figure}


\begin{figure}[t]
	\centering
    \caption[]{}
\end{figure}

\section{Discussion}


