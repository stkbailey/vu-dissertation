\chapter{Discussion}

The aim of this set of studies was to identify how attributes of network architecture allow for the flexibility required for skilled reading. A connectomics approach to reading illuminates -- not displaces -- previous neuroimaging research, much of which focused on localizing specific cognitive processes. Here, we asked how network architecture could inform our understanding of cognitive felxibility, especially in an integrative model such as reading. 

% Modularity
One consequence of modularity is evolvability. That is, a modular system is more capable of adapting to environmental circumstances because new modules can be added to the system without drastically altering the other modules \citep{Kashton2005}. Each one can work in parallel, sharing information when necessary, but without being too dependent on the success or failure of another system. This makes modularity not only efficient but also robust to damage. Cognition, too, has traditionally been considered to a large degree modular, with the pseudo-science of phrenology being the most extreme example, but more recent efforts coming from a deeper understanding of visual and sensory systems \citep{Barrett2006}. However, some higher-order cognitive functions such as working memory, attention and planning have not been localized to a discrete cortical area and are more likely to depend on a global ``workspace'' \citep{Dahaene1998}.

The cognitive processes in reading are also thought of in a modular sense: visual, auditory, semantic and motor processes can each be taught or assessed separately \citep{Cutting2009a}. Phonics instruction spends a great deal of time making individuals better at merging, specifically, the auditory and visual modules. Neuroimaging evidence from the past two decades suggets that this binding process localizes onto the left temporo-parietal junction \citep{}. 

Although it is still an area of investigation, one group investigated how modularity changed over the course of learning a novel task. Bassett and colleagues scanned participants at four timepoints while they were learning a new finger-tapping task: before training, early in training, midway through training and at the end of training \citep{Bassett2015}. The authors empirically defined a visual and motor module, and investigated changes to it throughout training. They found two important trends: first, the two modules became increasingly segegated throughout training and practice; and second, the involvement of non-module nodes such as those in subcortical systems was reduced over time. A separate study found that that modules are more likely to re-organize at early stages in the training process \citep{Bassett2011}. Taken together, the findings suggest a model in which the early stages of training require a high degree of cross-module communication, whereas later stages rely on more automated, modular processes (i.e. efficient and segregated processing). 


% Review of findings
In Chapter 1, we established that reading utilizes a variety of cognitive skills whose neural substrates are distributed throughout the brain. Graph theory has enabled anlysis...
Important properties... modules... hubs...
Individual differences not well understood...
Development through interactive specialization...

In Chapter 2, we analyzed the brain at rest in developing readers. We established a basis for our specific measures... tested different thresholds... next we compared modularity to individual differences...

In Chapter 3, we observed task-evoked changes to network architecture in reading... We found that reading increased measures of integration across the brain: decreased modularity, increased participation coefficient, etc... we also found that, even during reading, there was a strong correlation between modularity and reading skill, suggesting that ``flexibility'' was built on a strong core.

In Chapter 4, we addressed the question of variability in network architecture head-on by investigating two related processes - reading and listening - and comparing them to more simple, sensory-laden attention tasks and rest. We found that there was a common core of areas activated in listening and reading, and that there was also a common network backbone. More similarity in the two language conditions was indicative of better reading. But more similarity among the other conditions was also indicative... suggesting that stability among networks is actually the more important thing...

In Chapter 5, we replicated these findings with an independent task and also described the findings across development. We found that similarity between listening and reading - and indeed, all tasks -- _________ed over time. ...

\begin{table}[t]
	\renewcommand{\tabcolsep}{0.09cm}
	\centering
	\begin{tabular}{c|p{10cm}}
\toprule 
Figure & Key Finding \\ 
\midrule 
\ref{fig:ch2-global-glm-covariates-thresh} & Global modularity was the graph theory measurement most predictive of reading skill. \\ 
 \ref{fig:ch2-rsn-node-modularity-corr} & Modularity in the auditory and cingulo-opercular networks was anti-correlated with reading skill.	\\ 
\ref{fig:ch3-reading-connectome-activations} & Reading comprehension induces system-level increases in the ventral attention, visual, somatomotor (mouth) and default mode networks.	 \\ 
\ref{fig:ch3-comprehension-reorganization}  & Reading is especially characterized by decreased connectivity \textit{within} sensory, dorsal attention and default mode and increased connectivity \textit{between} many different RSNs.  \\ 
\ref{fig:ch3-modularity-reading-by-condition}  & The positive relationship between network modularity and reading persists during reading comprehension.  \\ 
\ref{fig:ch4-modality-graph-theory}  & Reading comprehension requires more more integration across networks than listening. \\ 
\ref{fig:ch4-modality-similarity-to-reading} & Better readers have greater similarity between their listening and reading networks.
\bottomrule 
\end{tabular}
	\caption[Key findings.]{Key findings in Studies 1 through 4.}
	\label{table:ch6-key-findings}
\end{table}

Overall, we have sought to use reading skill as a model for how individual differences in network architecture form a basis for individual differences in cognitive processing. We have combined inferences from several different methodological approaches, including task-based univariate activation analyses, resting-state network analysis, and the combination of the two. We believe we have made a contribution to our understanding... However, there is much left to be investigated. Below, we outline a few of these directions.

\section{Individualized parcellations}

A caveat with connectomics analyses, including those presented here, are that results for the modularity analyses are often based on RSN parcellations from previous literature (e.g. \citep{Power2011}) and are applied indiscriminately across the entire group. This allows for a common reference partition and more interpretable results, but it neglects the fact that there may be important differences between individuals in the optimal community partition for an individual. Even within a given method, there can be a very large number of alternative community parcellations that may differ from the maximum in only a very slight way \citep{Good2010}. Consensus-based clustering, such as that used in \citep{Power2011} can address some of these concerns, but is still essentially limited in its ability to allow variability of possible partitions in to the system. 

The problem may be especially acute in the context of development. Although this is an area of active investigation, as children mature, the pattern of connectivity changes significantly, with modules becoming more segregated and long-range RSNs such as the fronto-parietal network becoming more robustly connected \citep{Cao2016}. In cases where there are actual differences in the architecture, comparison to the same reference partition will result in one group being considered ``lower modularity'' when in fact they are ``different modularity''.  

One way to address this is to partition each subject's network individually

\section{Dynamic modeling of network architecture}

We could use dynamic modeling....




\section{Multi-modal investigations}


% Ventral attention network

\section{Conclusions}


