\chapter{Discussion}

A connectomics approach to reading illuminates -- not displaces -- previous neuroimaging research, much of which focused on localizing specific cognitive processes. Here, we asked how network architecture could inform our understanding of cognitive felxibility, especially in an integrative model such as reading. 

In Chapter 1, we established that reading utilizes a variety of cognitive skills whose neural substrates are distributed throughout the brain. Graph theory has enabled anlysis...
Important properties... modules... hubs...
Individual differences not well understood...
Development through interactive specialization...

In Chapter 2, we analyzed the brain at rest in developing readers. We established a basis for our specific measures... tested different thresholds... next we compared modularity to individual differences...

In Chapter 3, we observed task-evoked changes to network architecture in reading... We found that reading increased measures of integration across the brain: decreased modularity, increased participation coefficient, etc... we also found that, even during reading, there was a strong correlation between modularity and reading skill, suggesting that ``flexibility'' was built on a strong core.

In Chapter 4, we addressed the question of variability in network architecture head-on by investigating two related processes - reading and listening - and comparing them to more simple, sensory-laden attention tasks and rest. We found that there was a common core of areas activated in listening and reading, and that there was also a common network backbone. More similarity in the two language conditions was indicative of better reading. But more similarity among the other conditions was also indicative... suggesting that stability among networks is actually the more important thing...

In Chapter 5, we replicated these findings with an independent task and also described the findings across development. We found that similarity between listening and reading - and indeed, all tasks -- _________ed over time. ...

\begin{table}[t]
	\renewcommand{\tabcolsep}{0.09cm}
	\centering
	\begin{tabular}{c|p{10cm}}
\toprule 
Figure & Key Finding \\ 
\midrule 
\ref{fig:ch2-global-glm-covariates-thresh} & Global modularity was the graph theory measurement most predictive of reading skill. \\ 
 \ref{fig:ch2-rsn-node-modularity-corr} & Modularity in the auditory and cingulo-opercular networks was anti-correlated with reading skill.	\\ 
\ref{fig:ch3-reading-connectome-activations} & Reading comprehension induces system-level increases in the ventral attention, visual, somatomotor (mouth) and default mode networks.	 \\ 
\ref{fig:ch3-comprehension-reorganization}  & Reading is especially characterized by decreased connectivity \textit{within} sensory, dorsal attention and default mode and increased connectivity \textit{between} many different RSNs.  \\ 
\ref{fig:ch3-modularity-reading-by-condition}  & The positive relationship between network modularity and reading persists during reading comprehension.  \\ 
\ref{fig:ch4-modality-graph-theory}  & Reading comprehension requires more more integration across networks than listening. \\ 
\ref{fig:ch4-modality-similarity-to-reading} & Better readers have greater similarity between their listening and reading networks.
\bottomrule 
\end{tabular}
	\caption[Key findings.]{Key findings in Studies 1 through 4.}
	\label{table:ch6-key-findings}
\end{table}

Overall, we have sought to use reading skill as a model for how individual differences in network architecture form a basis for individual differences in cognitive processing. We have combined inferences from several different methodological approaches, including task-based univariate activation analyses, resting-state network analysis, and the combination of the two. We believe we have made a contribution to our understanding... However, there is much left to be investigated. Below, we outline a few of these directions.

\section{Individualized parcellations}

A caveat with connectomics analyses, including those presented here, are that results for the modularity analyses are often based on RSN parcellations from previous literature \citep{Power2011}. It is possible that in this sample of children, there were differences in network organization that resulted in a lower global modularity but were in fact due to differences in organization (e.g. multiple sub-modules of the default mode network). 


\section{Dynamic modeling of network architecture}

We could use dynamic modeling....




\section{Multi-modal investigations}


% Ventral attention network

\section{Conclusions}


