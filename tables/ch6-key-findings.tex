\begin{tabular}{cp{10cm}}
\toprule 
Figure & Key Finding \\ 
\midrule 
\multicolumn{2}{l}{\textsc{Study 1}} \\
\ref{fig:ch2-global-glm-covariates-thresh} & Global modularity was the graph theory measurement most predictive of reading skill. \\ 
\ref{fig:ch2-rsn-node-modularity-corr} & Modularity in the auditory and cingulo-opercular networks was anti-correlated with reading skill.	\\ 
\multicolumn{2}{l}{\textsc{Study 2}} \\
\ref{fig:ch3-reading-connectome-activations} & Reading comprehension induces system-level increases in the ventral attention, visual, somatomotor (mouth) and default mode networks.	 \\ 
\ref{fig:ch3-comprehension-reorganization}  & Reading is especially characterized by decreased connectivity \textit{within} sensory, dorsal attention and default mode and increased connectivity \textit{between} many different RSNs.  \\ 
\ref{fig:ch3-modularity-reading-by-condition}  & The positive relationship between network modularity and reading persists during reading comprehension.  \\ 
\multicolumn{2}{l}{\textsc{Study 3}} \\
\ref{fig:ch4-modality-graph-theory}  & Reading comprehension requires more more integration across networks than listening. \\ 
\ref{fig:ch4-modality-similarity-to-reading} & Better readers have greater similarity between their listening and reading networks.\\
\ref{fig:ch4-rsn-mean-similarity} & There is a relatively high level of mean similarity between network configurations across many tasks.\\
\multicolumn{2}{l}{\textsc{Study 4}} \\
\ref{fig:ch5-modularity-by-condition-3groups} & Modularity decreases with age but exhibits similar task-evoked changes in all groups. \\
\ref{fig:ch5-modularity-reading-corr-2groups} & Global modularity predicts reading skill in young readers better than adolescent ones. \\
\ref{fig:ch5-pairwise-iou-reading} & Adults showed less connectome similarity with their peers than children, although within-subject similarity remained the same.\\
\bottomrule 
\end{tabular}